\documentclass[a4paper, 12pt]{article}

\usepackage{url}
\usepackage{graphicx}
\usepackage{caption}
\usepackage[section]{placeins}
\usepackage{fixltx2e}
\usepackage[page]{appendix}

\usepackage{amsmath}
\usepackage{cleveref}

%for code(MATLAB in particular)
\usepackage{listings}
\usepackage{color} %red, green, blue, yellow, cyan, magenta, black, white
\definecolor{mygreen}{RGB}{28,172,0} % color values Red, Green, Blue
\definecolor{mylilas}{RGB}{170,55,241}

% Default fixed font does not support bold face
\DeclareFixedFont{\ttb}{T1}{txtt}{bx}{n}{12} % for bold
\DeclareFixedFont{\ttm}{T1}{txtt}{m}{n}{12}  % for normal

% Custom colors
\usepackage{color}
\definecolor{deepblue}{rgb}{0,0,0.5}
\definecolor{deepred}{rgb}{0.6,0,0}
\definecolor{deepgreen}{rgb}{0,0.5,0}

\lstset{
    language=Matlab,%
    %basicstyle=\color{red},
    breaklines=true,%
    morekeywords={matlab2tikz},
    keywordstyle=\color{blue},%
    morekeywords=[2]{1}, keywordstyle=[2]{\color{black}},
    identifierstyle=\color{black},%
    stringstyle=\color{mylilas},
    commentstyle=\color{mygreen},%
    showstringspaces=false,%without this there will be a symbol in the places where there is a space
    numbers=left,%
    numberstyle={\tiny \color{black}},% size of the numbers
    numbersep=9pt, % this defines how far the numbers are from the text
    emph=[1]{for,end,break},emphstyle=[1]\color{red}, %some words to emphasise
    %emph=[2]{word1,word2}, emphstyle=[2]{style},
}


\graphicspath{{./pictures/}}

\title{ECEN321 - Lab 4 \\
    Hypothesis Testing
}
\author{Joshua Benfell - 300433229}

\begin{document}
    \maketitle
    
    \section{Introduction}
        In this report a hypothesis test will be performed and analysed. The null hypothesis is that realisations of a poisson generated random variable is actually from a poisson distribution. This will be tested using a $\chi^2$ test. Consequently the alternate hypothesis is that the poisson random variable is not from a poisson distribution.

    \section{Method}
        To test this hypothesis, $M = 100$ inter-arrivals will be generated from a Poisson random variable. This will be done by transforming a uniform random variable into a poisson random variable. To do this we take the negtaive log of the uniform random variable and divide it by the poisson parameter $\lambda = 3$. From here, the cumulative sum is taken of the inter arrival times to find the times at which an arrival happens. 
    \section{Results}
    \section{Conclusion}

    % \Urlmuskip=0mu plus 1mu\relax
    % \bibliography{bibliography}
    % \bibliographystyle{IEEEtran}

    \begin{appendices}
    \end{appendices}
\end{document}